% The document class supplies options to control rendering of some standard
% features in the result.  The goal is for uniform style, so some attention 
% to detail is *vital* with all fields.  Each field (i.e., text inside the
% curly braces below, so the MEng text inside {MEng} for instance) should 
% take into account the following:
%
% - author name       should be formatted as "FirstName LastName"
%   (not "Initial LastName" for example),
% - supervisor name   should be formatted as "Title FirstName LastName"
%   (where Title is "Dr." or "Prof." for example),
% - degree programme  should be "BSc", "MEng", "MSci", "MSc" or "PhD",
% - dissertation title should be correctly capitalised (plus you can have
%   an optional sub-title if appropriate, or leave this field blank),
% - dissertation type should be formatted as one of the following:
%   * for the MEng degree programme either "enterprise" or "research" to
%     reflect the stream,
%   * for the MSc  degree programme "$X/Y/Z$" for a project deemed to be
%     X%, Y% and Z% of type I, II and III.
% - year              should be formatted as a 4-digit year of submission
%   (so 2014 rather than the academic year, say 2013/14 say).

\documentclass[ oneside,% the name of the author
                    author={George Herbert},
                % the degree programme: BSc, MEng, MSci or MSc.
                    degree={MSci},
                % the dissertation    title (which cannot be blank)
                     title={Video Diffusion Models for Climate Simulations},
                % the dissertation subtitle (which can    be blank)
                  subtitle={}]{dissertation}

\begin{document}

% =============================================================================

% This macro creates the standard UoB title page by using information drawn
% from the document class (meaning it is vital you select the correct degree 
% title and so on).

\maketitle

% After the title page (which is a special case in that it is not numbered)
% comes the front matter or preliminaries; this macro signals the start of
% such content, meaning the pages are numbered with Roman numerals.

\frontmatter


%\lstlistoflistings

% The following sections are part of the front matter, but are not generated
% automatically by LaTeX; the use of \chapter* means they are not numbered.

% -----------------------------------------------------------------------------

\chapter*{Abstract}

% {\bf A compulsory section, of at most 300 words} 
% \vspace{1cm} 

% \noindent
% This section should pr\'{e}cis the project context, aims and objectives,
% and main contributions (e.g., deliverables) and achievements; the same 
% section may be called an abstract elsewhere.  The goal is to ensure the 
% reader is clear about what the topic is, what you have done within this 
% topic, {\em and} what your view of the outcome is.

% The former aspects should be guided by your specification: essentially 
% this section is a (very) short version of what is typically the first 
% chapter. If your project is experimental in nature, this should include 
% a clear research hypothesis.  This will obviously differ significantly
% for each project, but an example might be as follows:

% \begin{quote}
% My research hypothesis is that a suitable genetic algorithm will yield
% more accurate results (when applied to the standard ACME data set) than 
% the algorithm proposed by Jones and Smith, while also executing in less
% time.
% \end{quote}

% \noindent
% The latter aspects should (ideally) be presented as a concise, factual 
% bullet point list.  Again the points will differ for each project, but 
% an might be as follows:

% \begin{quote}
% \noindent
% \begin{itemize}
% \item I spent $120$ hours collecting material on and learning about the 
%       Java garbage-collection sub-system. 
% \item I wrote a total of $5000$ lines of source code, comprising a Linux 
%       device driver for a robot (in C) and a GUI (in Java) that is 
%       used to control it.
% \item I designed a new algorithm for computing the non-linear mapping 
%       from A-space to B-space using a genetic algorithm, see page $17$.
% \item I implemented a version of the algorithm proposed by Jones and 
%       Smith in [6], see page $12$, corrected a mistake in it, and 
%       compared the results with several alternatives.
% \end{itemize}
% \end{quote}

% -----------------------------------------------------------------------------


\chapter*{Dedication and Acknowledgements}

% {\bf A compulsory section}
% \vspace{1cm} 

% \noindent
% It is common practice (although totally optional) to acknowledge any
% third-party advice, contribution or influence you have found useful
% during your work.  Examples include support from friends or family, 
% the input of your Supervisor and/or Advisor, external organisations 
% or persons who  have supplied resources of some kind (e.g., funding, 
% advice or time), and so on.


% -----------------------------------------------------------------------------

% This macro creates the standard UoB declaration; on the printed hard-copy,
% this must be physically signed by the author in the space indicated.

\makedecl



% -----------------------------------------------------------------------------

% LaTeX automatically generates a table of contents, plus associated lists 
% of figures and tables.  These are all compulsory parts of the dissertation.

\tableofcontents
\listoffigures
\listoftables

% -----------------------------------------------------------------------------



\chapter*{Ethics Statement}

% {\bf A compulsory section} 
% \vspace{1cm} 

% In almost every project, this will be one of the following statements:
%     \begin{itemize}
%         \item ``This project did not require ethical review, as determined by my supervisor, [fill in name]''; or
%         \item ``This project fits within the scope of ethics application 0026, as reviewed by my supervisor, [fill in name]''; or
%         \item ``An ethics application for this project was reviewed and approved by the faculty research ethics committee as application [fill in number]''.
%     \end{itemize}
    
% See Section 3.2 of the unit Handbook for more information. If something went wrong and none of those three statements apply, then you should instead explain what happened.


% -----------------------------------------------------------------------------

% \chapter*{Supporting Technologies}

% {\bf An optional section}
% \vspace{1cm} 

% \noindent
% This section should present a detailed summary, in bullet point form, 
% of any third-party resources (e.g., hardware and software components) 
% used during the project.  Use of such resources is always perfectly 
% acceptable: the goal of this section is simply to be clear about how
% and where they are used, so that a clear assessment of your work can
% result.  The content can focus on the project topic itself (rather,
% for example, than including ``I used \mbox{\LaTeX} to prepare my 
% dissertation''); an example is as follows:

% \begin{quote}
% \noindent
% \begin{itemize}
% \item I used the Java {\tt BigInteger} class to support my implementation 
%       of RSA.
% \item I used a parts of the OpenCV computer vision library to capture 
%       images from a camera, and for various standard operations (e.g., 
%       threshold, edge detection).
% \item I used an FPGA device supplied by the Department, and altered it 
%       to support an open-source UART core obtained from 
%       \url{http://opencores.org/}.
% \item The web-interface component of my system was implemented by 
%       extending the open-source WordPress software available from
%       \url{http://wordpress.org/}.
% \end{itemize}
% \end{quote}

% -----------------------------------------------------------------------------

\chapter*{Notation and Acronyms}

% {\bf An optional section}
% \vspace{1cm} 

% \noindent
% Any well written document will introduce notation and acronyms before
% their use, {\em even if} they are standard in some way: this ensures 
% any reader can understand the resulting self-contained content.  

% Said introduction can exist within the dissertation itself, wherever 
% that is appropriate.  For an acronym, this is typically achieved at 
% the first point of use via ``Advanced Encryption Standard (AES)'' or 
% similar, noting the capitalisation of relevant letters.  However, it 
% can be useful to include an additional, dedicated list at the start 
% of the dissertation; the advantage of doing so is that you cannot 
% mistakenly use an acronym before defining it.  A limited example is 
% as follows:

% \begin{quote}
% \noindent
% \begin{tabular}{lcl}
% AES                 &:     & Advanced Encryption Standard                                         \\
% DES                 &:     & Data Encryption Standard                                             \\
%                     &\vdots&                                                                      \\
% ${\mathcal H}( x )$ &:     & the Hamming weight of $x$                                            \\
% ${\mathbb  F}_q$    &:     & a finite field with $q$ elements                                     \\
% $x_i$               &:     & the $i$-th bit of some binary sequence $x$, st. $x_i \in \{ 0, 1 \}$ \\
% \end{tabular}
% \end{quote}


% =============================================================================

% After the front matter comes a number of chapters; under each chapter,
% sections, subsections and even subsubsections are permissible.  The
% pages in this part are numbered with Arabic numerals.  Note that:
%
% - A reference point can be marked using \label{XXX}, and then later
%   referred to via \ref{XXX}; for example Chapter\ref{chap:context}.
% - The chapters are presented here in one file; this can become hard
%   to manage.  An alternative is to save the content in seprate files
%   the use \input{XXX} to import it, which acts like the #include
%   directive in C.

\mainmatter


\chapter{Introduction}
\label{chap:context}

Diffusion models are latent variable models with latents $\mathbf{z}=\{\mathbf{z}_t\mid t\in[0,1]\}$. 

% =============================================================================

% Finally, after the main matter, the back matter is specified.  This is
% typically populated with just the bibliography.  LaTeX deals with these
% in one of two ways, namely
%
% - inline, which roughly means the author specifies entries using the 
%   \bibitem macro and typesets them manually, or
% - using BiBTeX, which means entries are contained in a separate file
%   (which is essentially a databased) then inported; this is the 
%   approach used below, with the databased being dissertation.bib.
%
% Either way, the each entry has a key (or identifier) which can be used
% in the main matter to cite it, e.g., \cite{X}, \cite[Chapter 2}{Y}.
%
% We would recommend using BiBTeX, since it guarantees a consistent referencing style 
% and since many sites (such as dblp) provide references in BiBTeX format. 
% However, note that by default, BiBTeX will ignore capital letters in article titles 
% to ensure consistency of style. This can lead to e.g. "NP-completeness" becoming
% "np-completeness". To avoid this, make sure any capital letters you want to preserve
% are enclosed in braces in the .bib, e.g. "{NP}-completeness".

\backmatter

\bibliography{dissertation}

% -----------------------------------------------------------------------------

% The dissertation concludes with a set of (optional) appendicies; these are 
% the same as chapters in a sense, but once signaled as being appendicies via
% the associated macro, LaTeX manages them appropriatly.

\appendix

\chapter{An Example Appendix}
\label{appx:example}

Content which is not central to, but may enhance the dissertation can be 
included in one or more appendices; examples include, but are not limited
to

\begin{itemize}
\item lengthy mathematical proofs, numerical or graphical results which 
      are summarised in the main body,
\item sample or example calculations, 
      and
\item results of user studies or questionnaires.
\end{itemize}

\noindent
Note that in line with most research conferences, the marking panel is not
obliged to read such appendices. The point of including them is to serve as
an additional reference if and only if the marker needs it in order to check
something in the main text. For example, the marker might check a program listing 
in an appendix if they think the description in the main dissertation is ambiguous.

% =============================================================================

\end{document}
